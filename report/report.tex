\input fontmac
\input mathmac

\def\Q#1{Q({\tt #1})}
\def\R#1{R({\tt #1})}
\def\A#1{A({\tt #1})}
\def\M#1{M({\tt #1})}
\def\S#1{S({\tt #1})}
\def\D#1{D({\tt #1})}
\def\F#1{F({\tt #1})}
\def\I#1{I({\tt #1})}
\def\E#1{E({\tt #1})}
\def\g{\underline{g}}
\def\x{\underline{x}}

\newcount\listcount
\def\resetnum{\global\listcount=1}
\def\numitem{\item{\romannumeral\listcount)}\global\advance\listcount by 1}

\maketitle{Finding regularity in Tlingit verb prefixes}{}{Marcel K. Goh}{30 April 2021}

\floattext5 \ninebf Abstract. \ninepoint
[Put an abstract here. Possibly think of a better title; the one above is something I came up in
five seconds that more or less describes the mumbo-jumbo my program produces. A more ambitious program
would have resulted in a title of ``Automatic segmentation and glossing of Tlingit verbs''. But alas my program
does no such thing.]

\advsect Introduction

[Probably an intro paragraph describing the Tlingit language
as a whole, where it fits into language families, and the current linguistic situation. Briefly mention dialectal
issues. Lure the reader in.
Mention related work, etc. References I will use are~\ref{corpus},
\ref{prefixes}, \ref{navajo2008}, \ref{tsuutina2017}, and \ref{viterbi}, and some others as well.]

\advsect Hidden and observed states

[Formalise the problem. Describe how the format of~\ref{prefixes} allows the tables to be decomposed
into a dictionary that maps
sequences of underlying morphemes to sequences of actual observed prefixes. We study the inverse problem.
Given a sequence of prefixes, how do we identify the original morphemes that they represent?
Describe what the tags mean and how
they will help us check when our rewrite approach has succeeded.]

\medskip
\boldlabel Strings and tags.
Formally, our program deals with the monoid of all strings over a finite alphabet $A$ (the set of all
letters in the Tlingit orthography), with the binary
operation of concatenation; strings in this set
will be written in a fixed-width typeface to distinguish them from the ordinary text of the paper. As an exception
to this convention, we use $\eps$ to denote the identity element (the empty string).
The data found in~\ref{prefixes}
is labelled, in the sense that every string has been partitioned into substrings,
each tagged with the type of prefix they represent. These tags are important to our program, so we will list
all of them here.
\medskip\resetnum
\numitem A {\it disjunct prefix} is tagged with $Q$. This may be a qualifier prefix, an incorporated noun
prefix, or an object prefix.
\smallskip
\numitem The {\it irrealis prefix} is tagged with $R$. The only two possibilities here are $\R u$ and $\R w$.
\smallskip
\numitem {\it Aspect prefixes} are tagged with $A$. An example is the perfective prefix, which can either
be $\A{wu}$ or $\A u$.
\smallskip
\numitem The {\it modality prefix} is tagged with $M$. The underlying form is always $\M\g$.
\smallskip
\numitem {\it Subject prefixes} are tagged with $S$.
\smallskip
\numitem The passive, antipassive, or middle voice is indicated by a {\it {\tt d}- prefix},
which is tagged with $D$.
\smallskip
\numitem Any of the {\it {\tt s}-, {\tt l}-, or {\tt sh}- prefixes} are classified under the $F$ tag, and
we only deal with the {\tt s} case, though it is noted in~\ref{prefixes} that the phonological patterns
are more or less analogous in the other two cases.
\smallskip
\numitem The stative {\it {\tt i}-prefix} is given the $I$ tag.
\smallskip
\numitem {\it Epenthesis} is indicated by the $E$ tag.
\medskip
Our program makes crucial use of these tags to detect when a sequence of observed prefixes corresponds exactly
to a sequence of morphemes. We denote by $A^*$ the set of all words formed from letters in the orthography, and
if the nine tags above are viewed as functions, we can let $B$ be the union of images of $A^*\setminus\{\eps\}$
under each of the nine functions above (an example of an element of $B$ is $\S{yi}$). Note that we never tag
the empty string $\eps$. Next, we consider the be the set of all nonempty sequences of elements of $B$,
denoted $B^+$; for instance, the sequence $\Q{ka}\S{\x}\D{d}\I{i}$ is an element of this set.
$B^+$ is a semigroup under
the operation of concatenation and can be made into a monoid $S$ by artificially adjoining an (untagged)
element $\eps$. This identity element will be important later when we start defining rewrite rules as functions
from $S$ to itself.

\medskip\boldlabel Hidden and observed sequences.
Although the data that the program uses is formatted as a table, it is simpler to think of it as a map (or
dictionary) from a set $H\subseteq S$ of sequences of underlying morphemes to a set $K\subseteq S$
of actual observed sequences of prefixes.

\advsect Rewrite rules

[The meat and potatoes of the paper. In this section I will describe all of the rewrite rules that my program
uses to match sequences of prefixes to sequences of underlying morphemes. A step-by-step sequence of rewrite
rules (dozens of them) will be elaborated, but it will also be fruitful to study what they mean individually
and combinatorially. For example, sometimes we will apply and then undo the same rule multiple times to
test it in combination with many other rules. Discuss the exponential blowup of such a scheme (for example,
if there are only 10 rewrites we can try on a particular sequence of prefixes, there are already $2^{10} = 1024$
different combinations we must try. Luckily, for any given word, there are not too many rewrites to try --- at
least, that I have seen so far. (My program is not yet done.)]

The following list describes a sequence of rewrite rules that the program applies, in order, to try to resolve
every sequence of prefixes. For $x,y\in S$, $x\neq\eps$, a
{\it rewrite rule} $x\to y$ can be viewed as a function from $S$ to itself that
replaces every instance of $x$ in a word $w\in S$ with $y$.

\medskip\resetnum
\numitem $\E{*}\to \eps$.
\medskip
The first rule removes all epenthesis from all strings in the data. This is something that we have the luxury
of doing because of our tagged data, and this step already resolves $45\%$ of all the entries found in the tables.
$$\Q{du}\R{w}\A{g}\M{\underline{g}}\S{du}\mapsto \Q{du}\A{g}\E{a}\M{\underline{x}}\S{du}$$
$$\A{g}\E{a}\M{\underline{x}} \to \R{w}\A{g}\M{\underline{g}}$$
$$\sum_{k=1}^{30} k!{30\choose k} \approx 7.21\times 10^{32}$$

\advsect Analysis of rewrite rules

[In this section, I want to analyse what the global sequence of rewrites actually means for individual strings.
It is expected that for a given string $w$, most rewrites do not actually affect $w$ directly, though
in certain cases we may have to try and undo the same rewrite multiple times to try different combinations.
It might also be interesting to see distributional data related to this (but I will have to code it up first).]

\advsect Grammar induction

[May not actually include, depending on how robust the above results actually are. If I end up not
doing this section, it will merge with the next section.
I was thinking of including a section about the general problem of grammar induction in formal language
theory and artificial intelligence.
My program does not come anywhere close to proper grammar induction, but I would
like to make comparisons, because the goals are largely aligned with ours. Our data just happens to be at
a smaller-scale and it also comes pre-tagged. In some sense, the way our data is set up makes the problem
a lot easier than the general problem of grammar induction. However, if we want to extend the program to
``real-world'' data such as verbs found in the text corpus, then we might have to fall back on tried-and-tested
approaches found in the literature (which may stumble because of the paucity of our data).]

\advsect Further directions

[Related to the grammar induction section. Reveal all of the shortcomings of our na\"\i ve program and
discuss what can be done to extend the algorithm to work on the verbs found in the corpus, not just
the nicely formatted verbs found in the prefix charts.]

\section References

[I have a convoluted Bash script that alphabetises, numbers, and outputs TeX for references. I know this
is not the preferred linguistics style, but I will keep it this way unless you would specificially like it
changed, since I don't currently have the time to modify my Bash script.]
\medskip

\bye
